\documentclass{ctexart}
\usepackage{graphicx}
\usepackage{geometry}
\usepackage{listings}
\usepackage{color}
\usepackage{pythonhighlight}

\title{产业结构与人口迁徙——新冠疫情潜在影响与次生灾害分析}
\author{曾\ \ \ 充\ 3180106183\ 计科1803 \\ 杨宇昊\ 3160105521\ 金融1601 \\ 张梁育\ 3180105674\ 电气1802}
\begin{document}
\begin{titlepage}
    \maketitle
\end{titlepage}
\tableofcontents
\newpage
\section{设计需求}

\subsection{设计背景}
新型冠状病毒肺炎(COVID-19,简称“新冠肺炎”)疫情肆虐全球多个国家,2020年3月11日,世界卫生组织 (WHO) 正式宣布将新冠肺炎列为全球性大流行病。在全球抗击新型冠状病毒疫情的过程中,产生了前所未有的大规模疫情数据,利用大数据分析技术和方法能够协助发现病毒传染源、监测疫情发展、调配救援物资,从而更好地进行疫情防控工作。可视分析作为大数据分析的重要方法,将数据智能处理、视觉表征和交互分析有机地结合,使机器智能和人类智慧深度融合、优势互补,为疫情防控中的分析、指挥和决策提供有效依据和指南。
\subsection{研究问题}
新型冠状病毒肺炎对我国经济发展造成了巨大冲击,在疫情发展的不同阶段,影响到了人口流动的趋势,同时对中国的经济产业结构造成了一次巨大的冲击。因此,我们小组利用可视分析技术,充分关联多源数据,来展示疫情走势与人口流动的变化趋势以及中国经济产业结构转型的状态。
\subsection{目标用户}
\begin{enumerate}
    \item 关注疫情影响的普通群众;
    \item 需要分析及评估疫情影响并制定相关防控措施的政府部门;
    \item 需要基于疫情影响做出合适决策,制定相关策略的企业与机构。
\end{enumerate}
\subsection{应用价值}
借助全球疫情走势、各国患者数量变化对比,人口流动、患者年龄分布和各省披露患者的分布和全国各地能够救治新冠医院分布的变化趋势,有助于分析疫情传播模式、比较全球各地主要是我国各地的传播差异、检测异常传播事件,制定传播管控策略,再辅以经济产业结构转型的状态,能够更好地评估疫情对国民经济、企业生产等方方面面带来的影响,防控复工复产困难、物资供需失衡等疫情次生灾害的发生。
\section{设计介绍}
\subsection{数据来源}

​		新型冠状病毒肺炎疫情(简称新冠疫情)是全球重大突发公共卫生事件,不仅对我国医疗卫生体系提出重大挑战,也对我国经济社会造成了重大冲击。CSMAR在疫情爆发后积极响应科研抗“疫”的理念,秉承为学术界提供一流研究数据的一贯精神,于国内率先推出了新冠疫情与经济研究数据库,助力学者开展经济、金融、管理等领域的新冠疫情相关研究工作。新冠疫情与经济研究数据库收录了疫情基本信息、人口流动、经济影响三部分数据。其中,疫情基本信息包括每日疫情数据、财政补助资金情况、确诊病例逗留地点分布、病患轨迹、医疗救治医院情况;人口流动包括省份及城市的迁入迁出人口比例;经济影响包括宏观经济、股市、上市公司的数据。
​		我们选取了 CSMAR 的
\begin{enumerate}
    \item 中国新冠肺炎确诊病患活动轨迹表
    \item 各地新冠肺炎医疗救治医院数量统计表
    \item 各城市迁出人口比例表(日)
    \item 各城市迁入人口比例表(日)
    \item 分省份国民生产总值表(年)数据表
    \item 国外新冠肺炎疫情动态表(日)
    \item 全球新冠肺炎确诊和死亡病例数统计表(日)
    \item 确诊病例小区分布表
\end{enumerate}

其中,具体数据条目和主要字段如下所示:

\begin{table}[!htbp]
    \begin{tabular}{p{100pt}p{230pt}p{60pt}p{130pt}}
        \hline
        数据名称                           & 主要字段                                                                 & 条目数量 \\ \hline
        中国新冠肺炎确诊患者活动轨迹表     & 省份名称、城市名称、县/区名称,发布时间、病患编号、性别、年龄            & 7146     \\
        各地新冠肺炎医疗救治医院数量统计表 & 地区代码、地区名称、医疗救治医院数量                                     & 385      \\
        各城市迁出人口比例表(日)         & 统计日期、迁出地区、隶属省份、迁出目的地                                 & 2432164  \\
        各城市迁入人口比例表(日)         & 统计日期、迁入地区、隶属省份、迁入来源地                                 & 2432164  \\
        分省份国民生产总值表(年)数据表   & 年度标识、省份名称、第一产业生产总值、第二产业生产总值、第三产业生产总值 & 2089     \\
        国外新冠肺炎疫情动态表(日)       & 统计日期、国家名称、确诊病例、新增确诊、死亡总数、新增死亡               & 69932    \\
        确诊病例小区分布表                 & 省份名称、城市名称、县/区名称、逗留地点、逗留人数                        & 8067     \\ \hline
    \end{tabular}
\end{table}
\newpage
\subsection{数据预处理}
我们综合使用 Python、 Excel 等多种工具或者语言对数据进行清洗,以便得到可以放置在可视化作品当中的JSON数据格式。这里列举部分数据预处理过程。
\subsubsection{医院数量分布数据}
\begin{python}
data = pd.read_excel("NCP_RegMedHosSta.xlsx")
data.drop(columns = "AreaCode",inplace=True)
data.columns = ['name','value']
data.drop(index=[0,1],inplace=True)
data['name'] = data['name'].str.rstrip('市') 
data.to_json("hospital.json",orient="table",index=False)
\end{python}
\subsubsection{迁入迁出人口数据}
\begin{python}
# 迁出数据
filenames = ["NCP_EmigrRatioD{}.xlsx".format(i) for i in ["","1","2","3","4"]]

data = dict()

for filename in filenames:
    data[filename] = pd.read_excel(filename)
    data[filename].columns =["统计日期","迁出地区","迁出地区代码","隶属省份","迁出目的地","迁出目的地代码","迁出人口比例"]
    data[filename].drop([0,1],inplace=True)
    data[filename].drop(columns=['迁出地区代码','隶属省份','迁出目的地代码'],inplace=True)
    data[filename]['统计日期'] = pd.to_datetime(data[filename]['统计日期'])
    data[filename]['统计日期'] = data[filename]['统计日期'].dt.date
    data[filename].sort_values(['统计日期','迁出地区','迁出人口比例'],ascending=[True,True,False],inplace=True)
    data[filename] = data[filename][~data[filename]['迁出目的地'].str.contains("省")]
    data[filename] = data[filename].groupby(['统计日期','迁出地区']).head(5)
    print(filename + " done")

result = pd.concat(data.values())
result.to_json("cleaned/emigrate.json",index=False)
print("succeed!")


# 迁入数据
filenames = ["NCP_ImmigrRatioD{}.xlsx".format(i) for i in ["","1","2","3","4","5"]]

data = dict()

for filename in filenames:
    data[filename] = pd.read_excel(filename)
    data[filename].columns =["统计日期","迁入地区","迁入地区代码","隶属省份","迁入来源地","迁入来源地代码","迁入来源地隶属省份","迁入人口比例"]
    data[filename].drop([0,1],inplace=True)
    data[filename].drop(columns=['迁入地区代码','隶属省份','迁入来源地代码',"迁入来源地隶属省份"],inplace=True)
    data[filename]['统计日期'] = pd.to_datetime(data[filename]['统计日期'])
    data[filename]['统计日期'] = data[filename]['统计日期'].dt.date
    data[filename].sort_values(['统计日期','迁入地区','迁入人口比例'],ascending=[True,True,False],inplace=True)
    data[filename] = data[filename][~data[filename]['迁入来源地'].str.contains("省")]
    data[filename] = data[filename].groupby(['统计日期','迁入地区']).head(5)
    print(filename + " done")

result = pd.concat(data.values())
result.to_excel("cleaned/immigrate.json",index=False)
result.reset_index(inplace=True)
result.to_json("cleaned/emigrate.json")
print("succeed!")
\end{python}
\subsubsection{分省份GDP数据}
\begin{python}
data = pd.read_excel("NCP_ProvincialGDP.xlsx")
data.drop(index=[0,1,2],inplace=True)
data.columns = ['年度标识', '省份编码', '省份名称', '地区生产总值-第一产业', '地区生产总值-第二产业', '地区生产总值-第三产业']
data.drop(columns=['省份编码'],inplace=True)
data.columns = ['year','province','one','two','three']
data = data.reindex(columns=['one','two','three','province','year']) 
data = data[data['province'] != '中国']
data.dropna(inplace=True)
data.sort_values(['year','province'],inplace=True)
data.to_json('provinceGDP2.json',orient='table',index=False)
file = open('provinceGDP2.json')
data = file.read()
c = eval(data)
years = [i for i in range(1952,2021)]
series_data = [[] for i in range(len(years))]
for i,year in enumerate(years):
    for j in c:
        j['year'] = int(j['year'])
        if j['year'] == year:
            series_data[i].append(list(j.values()))
d = open('test3.txt',mode='x',encoding='utf-8')
d.write(str(series_data))
\end{python}
\subsubsection{患者年龄分布数据}
\begin{python}
data = pd.read_excel("NCP_CnfrmdActTrack.xlsx")
data.drop(index=[0,1,2],inplace=True)
data.columns = ['省份代码', '省份名称', '城市代码', '城市名称', '县/区代码', '县/区名称','发布时间','病患编号','病患','性别','年龄']
data.drop(columns = ['省份代码','省份名称','城市代码','城市名称','县/区代码','县/区名称','病患编号','病患','性别'],inplace=True)
data['发布时间'] = pd.to_datetime(data['发布时间'])
data = data.assign(年龄段 = pd.cut(data['年龄'], bins=[i for i in range(0,110,20)],labels = ['儿童','青壮年','中年','中老年','老年']))
data.drop(columns=['年龄'],inplace=True)
data['发布时间'] = data['发布时间'].dt.date
a = data.value_counts()
a = a.sort_index()
a.to_json('test.json',orient='table')
\end{python}
\subsubsection{各地披露数据比例}
\begin{python}
data = pd.read_excel("NCP_CnfrmdCourtDisd.xlsx")
cleaned = []
cleaned.append(dict())
cleaned[0]['name'] = '中国'
cleaned[0]['children'] = []
cleaned[0]['value'] = 0

for index,row in data.iterrows():
    if row['省份名称'] is not None:
        cleaned[0]['value'] += row['逗留人数']
        for i in cleaned[0]['children']:
            if i['name'] == row['省份名称']:
                i['value'] += row['逗留人数']
                province = i
                break
        else:
            new_dict = dict()
            new_dict['name'] = row['省份名称']
            new_dict['value'] = row['逗留人数']
            new_dict['children'] = []
            cleaned[0]['children'].append(new_dict.copy())
            province = cleaned[0]['children'][-1]
    if row['城市名称'] is not None:
        for i in province['children']:
            if i['name'] == row['城市名称']:
                i['value'] += row['逗留人数']
                # city = i
                break
        else:
            new_dict = dict()
            new_dict['name'] = row['城市名称']
            new_dict['value'] = row['逗留人数']
            new_dict['children'] = []
            province['children'].append(new_dict.copy())
            city = province['children'][-1]
    if row['县/区名称'] is not None:
        for i in city['children']:
            if i['name'] == row['县/区名称']:
                i['value'] += row['逗留人数']
                xian = i
                break
        else:
            new_dict = dict()
            new_dict['name'] = row['县/区名称']
            new_dict['value'] = row['逗留人数']
            new_dict['children'] = []
            city['children'].append(new_dict.copy())
            xian = city['children'][-1]
    if row['逗留地点'] is not None:
        for i in xian['children']:
            if i['name'] == row['逗留地点']:
                i['value'] += row['逗留人数']
                break
        else:
            new_dict = dict()
            new_dict['name'] = row['逗留地点']
            new_dict['value'] = row['逗留人数']
            xian['children'].append(new_dict.copy())

with open('tmp.json',mode='x',encoding='utf-8') as f:
    f.write(str(cleaned))
\end{python}
\subsection{设计框架}
我们使用了如下的开发语言、开发框架和了开发工具。
\begin{enumerate}
    \item Echarts
    \item React
    \item Python
    \item Pandas
    \item Node.js
    \item VSCode
    \item Pycharm
    \item Excel
    \item Git
\end{enumerate}
\subsection{模块规划}
\subsubsection{人口迁徙与医院数量分布}
\subsubsection{各省份产业结构变化趋势}
\section{案例展示}
\subsection{主面板}

\subsection{人口迁徙与医院数量分布}


\subsection{各省份产业结构变化趋势}


\subsection{}

\end{document}